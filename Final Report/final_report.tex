\documentclass[a4paper,12pt]{report}
\usepackage{natbib}
\begin{document}
\title{Final report of the ast398 group's effort at the Giant Meterwave Radio Telescope to remove RFI and investigate the epoch of reionization}
\author{Joshua Albert \and Nidhi Bahavar \and Mark Kuiack \and Connie Lien \and Hans Nyugen}
\date{26 August, 2011}
\maketitle

\begin{abstract}
proposed format:
1 what our goals were
  - RFI search and destroy
  - EoR investigation
- use full names because we havn't introduced what RFI and EoR mean yet
2 what our techniques were
  - svd (is my explaination right?)
    3 what specifically did this locate? we were able to tell a wire man to go look at them. pomegranate farm? 
  - radio on while driving around technique
    4 what did this locate specifically? chicken farm?
5 were they removed? (is that all we did while there?)
6 what did we do for the EoR that is usefull? Do we have any values to report? perhaps a more streamlined procedure?

The first goal of this project was to devise a standard procedure to locate and eliminate radio frequency interference, within the near vicinity of the Giant Meterwave Radio Telescope, and make the procedure streamlined so that eventually others could do it, possibly remotely, in the future. The second aim was to continue the data collection required for others\citep{others} investigate the epoch of reionization. The radio frequency was located with several techniques, which are, doing singular value decomposition on visibility ranges, called modes, and selecting the top 20 eigenvalues.
\end{abstract}

\section{Introduction}
\section{Definition of Terms}
Radio frequency interference, hereafter refered to as RFI. Epoch of reionization, hereafter refered to as EoR.

\section{body}
The foundations of the rigorous study of \emph{analysis}
were laid in the nineteenth century, notably by the
mathematicians Cauchy and Weierstrass. Central to the
study of this subject are the formal definitions of
\emph{limits} and \emph{continuity}.

Let $D$ be a subset of $\bf R$ and let
$f \colon D \to \mathbf{R}$ be a real-valued function on
$D$. The function $f$ is said to be \emph{continuous} on
$D$ if, for all $\epsilon > 0$ and for all $x \in D$,
there exists some $\delta > 0$ (which may depend on $x$)
such that if $y \in D$ satisfies
\[ |y - x| < \delta \]
then
\[ |f(y) - f(x)| < \epsilon. \]

One may readily verify that if $f$ and $g$ are continuous
functions on $D$ then the functions $f+g$, $f-g$ and
$f.g$ are continuous. If in addition $g$ is everywhere
non-zero then $f/g$ is continuous.

\end{document}
